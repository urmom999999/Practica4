\documentclass{article}
\usepackage[utf8]{inputenc}
\usepackage{graphicx,float}
\usepackage{hyperref}
\usepackage{listings}
\usepackage{verbatim}

\graphicspath{{imagenes/}}
\title{Programación de Servicios y Procesos: Práctica de Docker Local}
\author{Álvaro Del Valle Fernández}

\begin{document}
\section{Introducción}
En esta practica completaré 4 ejercicios diferentes en Odoo usando como referencia los archivos del repositorio de git entregado.\\
Antes de comenzar con los ejercicios creare un odoo completamente nuevo en la version 18, realizando mas ajustes para no tener problemas constantes de nuevo.
\begin{verbatim}
 docker-compose up -d
 \end{verbatim}
    \begin{figure}[H]
    \centering
    \includegraphics[width=4in]{img1.png}
\end{figure}
Credenciales dentro de Odoo:
    \begin{figure}[H]
    \centering
    \includegraphics[width=4in]{img2.png}
\end{figure}
Cada vez que tengo que probar los cambios realizados, debo usar el comando en docker: 
\begin{verbatim}
 docker-compose restart odoo
 docker-compose up -d
 \end{verbatim}
Tras ello debo hacer click en los tres puntos del modulo indicado y darle a "Actualizar"
\begin{figure}[H]
    \centering
    \includegraphics[width=4in]{img8.png}
\end{figure}
Esto puede que me de una notificacion de error, normalmente sin indicar el origen del error.\\
\begin{figure}[H]
    \centering
    \includegraphics[width=4in]{img9.png}
\end{figure}
Los logs de Docker son mi principal fuente de informacion para detectar estos errores y arreglarlos.
\begin{figure}[H]
    \centering
    \includegraphics[width=4in]{img8a.png}
\end{figure}





\section{01: Gestión de Tareas Mejorada}
    \subsection{Objetivo}
Para realizar este ejercicio debo añadir el datetime y mostrarla en una vista nueva tipo Kanban, junto a otra vista tipo calendario, para ello debo crear dos views nuevas.
Kanban es un tipo de vista que muestra la informacion en columnas mediante tarjetas, ideal para esta lista de tareas. \\
    \subsection{Implementación}
     Usando de referencia el archivo view de git, junto guias para aprender la estructura Kanban, cree las dos vistas pedidas.\\
    La vista Kanban:
\begin{verbatim}
<?xml version="1.0" encoding="utf-8"?>
<odoo>
  <data>
    <record id="view_lista_tareas_kanban" model="ir.ui.view">
      <field name="name">lista.tareas.kanban</field>
      <field name="model">lista_tareas.lista</field>
      <field name="arch" type="xml">
        <kanban default_group_by="estado">
          <field name="tarea"/>
          <field name="prioridad"/>
          <field name="urgente"/>
          <field name="realizada"/>
          <field name="fecha_asignada"/>
          <field name="estado"/>
          
          <templates>
            <t t-name="kanban-box">
              <div class="oe_kanban_card oe_kanban_global_click">
                <div>
                  <strong><field name="tarea"/></strong>
                </div>
                <div>
                  Fecha: <field name="fecha_asignada"/>
                </div>
                <div>
                  Prioridad: <field name="prioridad"/>
                </div>
                <div>
                  <field name="realizada"/> Realizada
                </div>
              </div>
            </t>
          </templates>
        </kanban>
      </field>
    </record>
  </data>
</odoo>
\end{verbatim}
Y la vista del calendario:
\begin{verbatim}
<?xml version="1.0" encoding="utf-8"?>
<odoo>
  <data>

    <record id="view_lista_tareas_calendar" model="ir.ui.view">
      <field name="name">lista.tareas.calendar</field>
      <field name="model">lista_tareas.lista</field>
      <field name="arch" type="xml">
        <calendar string="Tareas por Fecha"
                  date_start="fecha_asignada"
                  date_stop="fecha_asignada"
                  color="prioridad"
                  mode="month"
                  event_open_popup="true">
          <field name="tarea"/>
          <field name="prioridad"/>
          <field name="estado"/>
          <field name="urgente"/>
          <field name="realizada"/>
        </calendar>
      </field>
    </record>
  </data>
</odoo>
\end{verbatim}
Elementos cambiados en manifest para cargar las vistas propiamente:
\begin{verbatim}
    'data': [
        # Permisos de acceso al modelo
        'security/ir.model.access.csv',

        # Archivo XML con las vistas, menús y acciones del modelo
        'views/views.xml',
        'views/kanban.xml', 
        'views/calendario.xml',
    ]
\end{verbatim}
Tambien hay que realizar cambios en views devido a las dos vistas nuevas, añadiendo calendar y kanban al viewMode:
\begin{verbatim}
        <record model="ir.actions.act_window" id="action_lista_tareas">
      <!-- Título visible de la ventana -->
      <field name="name">Listado de tareas</field>

      <!-- Modelo al que se refiere la acción -->
      <field name="res_model">lista_tareas.lista</field>

      <!-- Tipo de vistas que se mostrarán:
           list → vista en tabla
           form → vista en formulario -->
      <field name="view_mode">kanban,calendar,list,form</field>
\end{verbatim}
Al enlazar el manifest junto el views permite mostrar correctamente las actualizaciones realizadas. Ejemplo de dos creadas ya con las modificaciones.
\begin{figure}[H]
    \centering
    \includegraphics[width=4in]{img10.png}
\end{figure}
Vista del menu al seleccionar "Nuevo"
\begin{figure}[H]
    \centering
    \includegraphics[width=4in]{img11.png}
\end{figure}
Ahora tenemos la opcion nueva del calendario, en la cual, si le damos click nos muestra la view nueva, aqui podemos seleccionar el dia de entrega en el calendario.
\begin{figure}[H]
    \centering
    \includegraphics[width=4in]{img12.png}
\end{figure}
Con esto el ejercicio esta completado, me encontré con numerosos errores, el principal siendo que al realizar cambios no estaba actualizando correctamente docker ni la aplicacion
por lo que no podia ver mi progreso. Odoo me indico multiples veces de errores inesperados pero sin dar un codigo de error claro, lo que me hizo perder mucho tiempo. El principal
 error fue no añadir al manifest las views correctamente, nombrandola calendar, calendario y viewcalendario.xml pensando que era otro elemento el que unia las views entre si.\\
 Otro error fue un field name incorrecto que use de views.xml, el cual era llamado pero pero al no existir en models.py no permitia al programa funcionar.

 

\section{02: Biblioteca de Cómics Ampliada}
    \subsection{Objetivo}

    \subsection{Implementación}

    \subsubsection{Modelo de Datos}

\section{03: Gestión Hospitalaria}
    \subsection{Objetivo}

    \subsection{Implementación}

    \subsubsection{Modelo de Datos}

\section{04: Gestión de Ciclos Formativos}
    \subsection{Objetivo}

    \subsection{Implementación}
Error en el ir.model.access.csv que produjo que no lo detectase sin dar un error claro.\\
Multiples errores con las horas\\
Error continuo que no se solucionó hasta que seleccioné Delete y borre y reinstale la App de nuevo, tampoco sin indicar desde Odoo la causa del problema.
\begin{figure}[H]
    \centering
    \includegraphics[width=4in]{img42.png}
\end{figure}
    \subsubsection{Modelo de Datos}
    
\end{document}