\documentclass{article}
\usepackage[utf8]{inputenc}
\usepackage{graphicx,float}
\usepackage{hyperref}
\usepackage{listings}
\usepackage{verbatim}

\graphicspath{{imagenes/}}
\title{Programación de Servicios y Procesos: Práctica de Docker Local}
\author{Álvaro Del Valle Fernández}

\begin{document}
\section{Introducción}
\begin{verbatim}
 docker-compose up -d
 \end{verbatim}

\section{01: Gestión de Tareas Mejorada}
    \subsection{Objetivo}

    \subsection{Implementación}
    Codigo de la Lista de tareas, model proporcionado en git:
\begin{verbatim}
    # -*- coding: utf-8 -*-

# Importamos los módulos necesarios de Odoo para definir modelos
from odoo import models, fields, api

# Creamos nuestro modelo de datos principal.
# Todos los modelos de Odoo deben heredar de models.Model
class ListaTareas(models.Model):  # Buenas prácticas: nombres de clase en PascalCase (MayúsculaInicial)
    
    # Nombre técnico del modelo. Es como Odoo lo guarda internamente en la base de datos
    _name = 'lista_tareas.lista'

    # Descripción que aparece en la documentación y ayuda
    _description = 'Modelo de la lista de tareas'

    # Indica qué campo se mostrará por defecto como nombre del registro (en vistas y menús desplegables)
    _rec_name = "tarea"

    # Definimos los campos (atributos) que tendrá cada registro de este modelo:

    # Campo de tipo texto (cadena). Será el nombre de la tarea.
    tarea = fields.Char(string="Tarea")

    # Campo de tipo entero. Se usará para indicar la prioridad (ej: 1 a 100)
    prioridad = fields.Integer(string="Prioridad")

    # Campo calculado de tipo booleano. Será True si la prioridad > 10
    # compute indica el método que lo calcula
    # store=True guarda el valor en la base de datos para poder filtrar y ordenar por él
    urgente = fields.Boolean(string="Urgente", compute="_value_urgente", store=True)

    # Campo booleano normal. Será marcado si la tarea ya se realizó.
    realizada = fields.Boolean(string="Realizada")

    # -------------------------------
    # MÉTODO COMPUTADO
    # -------------------------------
    # Este método se ejecuta cada vez que cambie el campo 'prioridad'
    @api.depends('prioridad')
    def _value_urgente(self):
        for record in self:
            # Si la prioridad es mayor que 10, se considera urgente
            record.urgente = record.prioridad > 10
\end{verbatim}
Para realizar este ejercicio debo añadir el datetime y mostrarla en una vista nueva tipo Kanban, junto a otra vista tipo calendario, para ello debo crear dos views nuevas.
Kanban es un tipo de vista que muestra la informacion en columnas mediante tarjetas, ideal para esta lista de tareas.

    \subsubsection{Modelo de Datos}

\section{02: Biblioteca de Cómics Ampliada}
    \subsection{Objetivo}

    \subsection{Implementación}

    \subsubsection{Modelo de Datos}

\section{03: Gestión Hospitalaria}
    \subsection{Objetivo}

    \subsection{Implementación}

    \subsubsection{Modelo de Datos}

\section{04: Gestión de Ciclos Formativos}
    \subsection{Objetivo}

    \subsection{Implementación}

    \subsubsection{Modelo de Datos}
    
\end{document}